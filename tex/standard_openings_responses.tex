\chapter{Standard openings and responses}
\label{ch:standard_openings_responses}

\section{Opening structure}

Some text explaining when we open\ldots

%A minimum opening hand should have 12 HCPs, or have the total of the high card points and the
%length in the two longest suits reach at least 19. Obviously judgement should be used, and third
%and fourth seat might open with less than first and second seat. Favourable vulnerability may also
%incentivise slightly lighter openings. Note that at EBU level 4, it is illegal to agree to open
%(presumably even in third/fourth seat) with a hand that does not satisfy the previously mentioned
%rule of 19.
%
%Personally, I apply the following rules (noting that a 12 HCP hand would almost always be opened).
%For unbalanced openings:
%
%\begin{itemize}
%    \item Rule of 20: points in hand plus length in longest two suits sums to at least 20
%    \item Hand has at most 7 losing tricks
%    \item Hand doesn't have terrible honours (e.g. honours placed separately in short suits, lots
%        of ``quacks'', few 10s and 9s)
%\end{itemize}
%
%For a balanced hand:
%
%\begin{itemize}
%    \item At least 11 points, normally 12
%    \item If 11 points, some intermediary values and likely a 5-card suit
%\end{itemize}
%
%In third or fourth seat, I relax the rule of 20, and would potentially open a hand with less,
%particularly if my length in spades plus high card points reached 15.

\begin{table}[H]
\begin{tabular}{l l}
    \ta 1\bCs & 12-21 points with 3+ clubs (better minor)\\
    \tb 1\bDs & 12-21 points with 3+ diamonds (better minor)\\
    \ta 1\bHs & 12-21 points with 5+ hearts\\
    \tb 1\bSs & 12-21 points with 5+ spades\\
    \ta 1\bNs & 15-17 points balanced\\
    \tb 2\bCs & a) 22+ points balanced\\\te
    \tb       & b) Unbalanced game forcing hand\\
    \ta 2\bD/\bH/\bSs & Weak two, 6 card suit, 5-10 HCPs\\
    \tb 2\bNs  & 20-21 points, (semi)balanced\\
    \ta 3\bany & Pre-emptive, 7 cards, 4-9 HCPs\\
    \tb 3\bNs  & Weak gambling: 7+ card minor with AKQ, no outside stoppers\\
    \ta 4\bany & 8+ card suit\\
    \tb 4\bNs  & Specific ace ask\\
    \ta 5\bmin & Pre-emptive\\
\end{tabular}
\end{table}

\subsubsection{Canapé}

With both diamonds and clubs (at least 5-4), always open diamonds. This is discussed further in
Section~\ref{sec:canape}.

\section{1\bmins opening}

1\bmins opening shows at least 3 cards in the suit, and denies a 5 card major. At the one level,
bid 4-card suits up the line.

\begin{table}[H]
\begin{tabular}{l l l}
    \tb        & \underline{Over 1\bC}              & \underline{Over 1\bD}\\[0.1cm]
    \ta 1\bDs  & 6+ points, 4+ diamonds, forcing    & -\\
    \tb 1\bHs  & \multicolumn{2}{c}{6+ points, 4+ hearts, forcing}\\
    \ta 1\bSs  & \multicolumn{2}{c}{6+ points, 4+ spades, forcing}\\
    \tb 1\bNs  & \multicolumn{2}{c}{6-10 points, no 4 card major}\\
    \ta 2\bCs  & 11+ points, 5(4)+ clubs            & 10+ points, 4+ clubs, no 4 card major\\
    \tb 2\bDs  & 17+ points, 6+ diamonds            & 11+ points, 5(4)+ diamonds\\
    \ta 2\bmaj & \multicolumn{2}{c}{17+ points, 6+ card suit}\\
    \tb 2\bNs  & \multicolumn{2}{c}{11-12 points, no 4 card major, no 5 card support}\\
    \ta 3\bCs  & 0-9 points, 5+ clubs               & 17+ points, 6+ clubs\\
    \tb 3\bDs  & Splinter, 5 card support           & 0-9 points, 5+ clubs\\
    \ta 3\bmaj & \multicolumn{2}{c}{Splinter, 5 card support}\\
    \tb 3\bNs  & \multicolumn{2}{c}{To play, 13-15 points, no 4 card major}\\
    \ta 4\bCs  & Pre-emptive raise                  & Splinter, 5+ diamonds\\
    \tb 4\bDs  & -                                  & Pre-emptive raise\\
    \ta 4\bNs  & \multicolumn{2}{c}{RKCB}\\
    \tb 5\bCs  & To play                            & -\\
    \ta 5\bDs  & -                                  & To play\\
\end{tabular}
\end{table}

\subsection{Forcing minor suit raises}

A direct raise in the minor suit shows 10+ points and 5 card support. It may also be used with 4
cards if there is no better option (bid \bNs with a flat hand) and the suit is good. Opener's
options after a raise in diamonds are as follows:

\begin{table}[H]
\begin{tabular}{l l}
    \ta 2\bmajs & Shows additional values and a stopper in the suit\\
    \tb 2\bNs   & Minimum balanced with 2 diamonds\\
    \tb         &     3\bC/\bmajs shows a stopper and is game forcing\\
    \tb         &     3\bDs       shows minimum values with a holding unsuitable for no trump\\
    \ta 3\bCs   & Shows additional values and a stopper in clubs\\
    \tb 3\bDs   & Minimum with 4 diamonds\\
    \ta 3\bmajs & Splinter\\
    \tb 3\bNs   & Values for game with a 3 card diamond suit\\
    \ta 4\bCs   & Splinter\\
    \tb 5\bDs   & Strong unbalanced hand without stoppers\\
\end{tabular}
\end{table}

\section{1\bmajs opening}

\begin{table}[H]
\begin{tabular}{l l l}
    \tb        & \underline{Over 1\bH} & \underline{Over 1\bS}\\[0.1cm]
    \ta 1\bSs  & 6+ points 4+ spades, forcing   & -\\
    \tb 1\bNs  & \multicolumn{2}{c}{6-10 points, 3− spades}\\
    \ta 2\bmin & \multicolumn{2}{c}{10+ points, 4 card suit} \\
    \tb 2\bHs  & 6-9 points, 3 hearts           & 10+ points, 5+ hearts, forcing\\
    \ta 2\bSs  & 17+ points, 6+ spades          & 6-9 points, 3 spades\\
    \tb 2\bNs  & \multicolumn{2}{c}{Jacoby (4+ card support, game forcing)}\\
    \ta 3\bCs  & \multicolumn{2}{c}{Bergen (6-9 points, 4+ card support)}\\
    \tb 3\bDs  & \multicolumn{2}{c}{Bergen (10-12 points, 4+ card support)}\\
    \ta 3\bHs  & 0-8 points, 4 hearts           & Splinter, 4+ spades\\
    \tb 3\bSs  & Splinter, 4+ hearts            & 0-8 points, 4 spades\\
    \ta 3\bNs  & \multicolumn{2}{c}{13-15 points balanced, to play}\\
    \tb 4\bmin & \multicolumn{2}{c}{Splinter, 4+ card support}\\
    \ta 4\bHs  & Pre-emptive raise              & -\\
    \tb 4\bSs  & -                              & Pre-emptive raise\\
    \ta 4\bNs  & \multicolumn{2}{c}{RKCB}\\
\end{tabular}
\end{table}

With 3 card support and a game forcing or invitational hand, first change suit and then bid
partner's suit to the appropriate level.

\subsubsection*{After interference}

Do something.

\subsection{Jacoby 2\bN}

This shows 4 card support for opener's major without a singleton or void (use splinters for that),
and is unlimited in strength (at least forcing to game). Opener's responses are as follows after
1\bHs - 2\bN:

\begin{table}[H]
\begin{tabular}{l l}
    \ta 3\bmin/\bSs & Singleton or void in the suit bid\\
    \tb 3\bHs       & Very strong, no shortness, 17+ points\\
    \ta 3\bNs       & Strong, no shortness, 15-16 points\\
    \tb 4\bmin/\bSs & Side 5 card suit with a top honour, or 4 card suit with 2 top honours\\
    \ta 4\bHs       & Minimum (12-14 points)\\
\end{tabular}
\end{table}

\subsection{Bergen raises}

A bid by opener of 3 in the major is a sign-off, new suits are invitational and trial bids.

\subsection{Drury}

Over a third or fourth seat 1\bmajs opening, 2\bCs no longer shows natural clubs but instead a game
invitational hand with 3 card support for opener's major. The bid asks opener to clarify their
hand strength as follows (assume the sequence 1\bSs - 2\bC):

\begin{table}[H]
\begin{tabular}{l l}
    \ta 2\bDs  & Shows a good opening hand with 14+ points, 5+\bmajs and 7− losers\\
    \tb 2\bSs  & Shows a minimal opening hand\\
    \ta 2\bHs  & Shows a good opening hand with 4+ hearts\\
\end{tabular}
\end{table}

With long clubs, responder must lie, bidding Drury first and then 3\bC. This shows a maximum passed
hand.

\section{1\bNs opening}

% TODO
%     What are the 3-level responses? (natural, splinter, puppet Stayman)
%     1N - 2C - 2H - 3D? (trial, splinter, natural)
%     1N - 2C - 2D - 2S? (weak with both majors)
%     1N - 2D - 2H - 4C? (splinter)

\begin{table}[H]
\begin{tabular}{l l}
    \ta 2\bCs     & Non-promissory Stayman\\
    \tb 2\bD/\bHs & Transfers to the majors, 5+ hearts/spades\\
    \ta 2\bSs     & Transfer to clubs (6+ clubs)\\
    \tb 2\bNs     & Transfer to diamonds (6+ diamonds)\\
    \ta 3\bany    & -\\
    \tb 3\bNs     & To play\\
    \ta 4\bCs     & Gerber\\
    \tb 4\bD/\bHs & Texas transfers, 6+ hearts/spades, 9-12 points\\
    \ta 4\bSs     & Quantitative invite to 6\bN, has a 5-card minor\\
    \tb 4\bNs     & Quantitative invite to 6\bN, no 5-card minor\\
\end{tabular}
\end{table}

\subsection{Stayman}

Responses are:

\begin{itemize}
    \item 2\bDs denies a four card major
    \item 2\bHs shows four hearts but does not deny four spades
    \item 2\bSs shows four spades and denies four hearts
\end{itemize}

\subsubsection{Responder's rebid}

\begin{itemize}
    \item A rebid of 2\bNs is a balanced invite without support in opener's major if one was bid.
    \item Raising the major shows 4-card support and is game invitational
    \item A bid of 3\bNs or 4 of opener's major is to play
    \item 1\bNs - 2\bCs - 2\bHs - \textbf{2\bS}: Invitational with 4 spades
    \item 1\bNs - 2\bCs - 2\bDs - \textbf{3\bmaj}: Smolen, game forcing with 4 cards in the bid
        major and 5 cards in the other major
        \begin{itemize}
            \item Opener bids the major with the best fit at the lowest level
            \item With no fit, opener bids 3\bN (after this, 4\bD/\bHs is a transfer to 4\bH/\bSs,
                showing 6 cards in the suit)
        \end{itemize}
\end{itemize}

\subsection{4-way transfers}

Opener should normally complete the transfer, but may superaccept with a maximum and 4-card support.
A new suit by responder is game forcing and natural.

\subsubsection{Major suit transfers}

Superaccepts are shown by completing the transfer at the 3 level.

Raising the transferred suit shows a sixth card and is invitational (or, if at the game level, to
play). Bidding 2\bNs shows a balanced invitational hand, and bidding 3\bNs shows a balanced choice
of games.

\subsubsection{Minor suit transfers}

Superaccepts are shown by bidding the ``intermediate'' suit, i.e. 1\bNs - 2\bSs - \textbf{2\bNs}
(superaccept).

\section{Strong openings}

\subsection{2\bCs opening}

2\bCs should be opened on any 22+ point count, or any hand that can see game in hand.

\begin{table}[H]
\begin{tabular}{l l}
    \ta 2\bDs & Waiting, could have 8+ points\\
    \tb 2\bmajs & 2 top honours in a 5 card major suit\\
    \ta 2\bNs & 10 points, no other bid\\
    \tb 3\bmins & Very good suit or hand\\
\end{tabular}
\end{table}

\subsubsection{Opener's rebid}

With a balanced hand, opener will rebid 2\bNs (unless 25 points, in which case a higher level of no
trumps), or a 5 card suit. A rebid of 2\bNs acts the same as an opening bid of 2\bNs in terms of
responder's bids. Any other bid is forcing to game. Suits are bid naturally until a fit is found or
game is bid.

Responder should support suits at the lowest possible level when slightly stronger, and bid game if
very weak.

After responder's 2\bNs bid opener can bid 3\bNs with a minimum balanced hand, or otherwise bids
suits naturally.

\subsection{2\bNs opening and 2\bNs rebid after 2\bC opener}

Opened with 20-21 points semi-balanced, responses over 2\bNs are similar to those over 1\bN.

\begin{table}[H]
\begin{tabular}{l l}
    \ta 2\bCs     & Stayman\\
    \tb 2\bD/\bHs & Transfers to majors\\
    \ta 2\bSs     & Minor suit slam try\\
    \tb 3\bNs     & To play\\
    \ta 4\bCs     & Gerber\\
    \tb 4\bD/\bHs & Texas transfers to majors\\
    \ta 4\bS      & Quantitative with a 5-card minor\\
    \tb 4\bNs     & Quantitative invite\\
\end{tabular}
\end{table}

\subsubsection{3\bSs bid}

This shows a minor or minors and a slam try. Opener bids over 3\bNs with potential interest in both
minors, otherwise bidding 3\bN. Over 3\bN, responder bids 4\bmins to show a single suited slam try,
or with both minors, bids a major suit to show a shortness.

\section{2\bD/\bH/\bSs openings (weak twos)}

To qualify for this bid, opener should have 5-10 points and a good six card suit containing at least
one of the top three honours (as a guide). Four card side suits are allowed, but if a hand passes
the rule of 20 it should be opened at the one level. Some hands are too distributional to be opened
with a weak two (e.g. 6520). In this case, it is probably better to pass and enter the auction
later. Responses go along the lines of RONF (raise only non-forcing).

\begin{table}[H]
\begin{tabular}{l l}
    \ta New suit & 16+ points, 5+ card suit, forcing\\
    \tb 2\bNs    & Ogust\\
    \ta Raise    & Pre-emptive\\
    \tb 3\bNs    & To play\\
\end{tabular}
\end{table}

\subsection{Change of suit}

\subsection{Ogust}

\section{Other openings}

\subsection{Pre-emptive openings}

\subsection{Gambling 3\bN}

\subsection{Specific ace ask 4\bN}

